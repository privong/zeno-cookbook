\documentclass[11pt,letterpaper]{article}

\usepackage{hyperref,natbib}
\usepackage[left=1in,top=1in,right=1in,bottom=1in]{geometry}

\begin{document}

\title{Zeno N-body/SPH Cookbook}
\author{George C. Privon\\ \href{mailto:gcp8y@virginia.edu}{gcp8y@virginia.edu}}

\maketitle

%-------------------------------------------------------------------------------
\section{Introduction}
\label{sec:Introduction}

Zeno is a flexible N-body and Smoothed Particle Hydrodynamics designed for simulations of collisionless gravitational systems. It is generally applied to the question of galaxy interactions and mergers. The software was written and is maintained by Prof. Joshua Barnes\footnote{\url{http://www.ifa.hawaii.edu/faculty/barnes/}} at the Institute for Astronomy, University of Hawaii. The code is released under the GNU General Public License.

The routines available with Zeno are powerful yet specifc; in a shell environment, these commands can be chained together via UNIX pipes and/or Makefiles to build powerful analysis pipelines.

This cookbook provides an overview of the Zeno environment and how to get it running (this section), the main algorithms used (Sections \ref{sec:Nbody} \& \ref{sec:SPH}), an overview of creating simple initial conditions (Section \ref{sec:IC}), and examples of analysis chains in the Zeno environment (Section \ref{sec:Analysis}).

This section begins with a quick guide to installing Zeno (\ref{ssec:installing}) and a description of the naming convention for Zeno programs (\ref{ssec:programs}).

\subsection{Installing Zeno}
\label{ssec:installing}

The Zeno source code is freely distributed and is available online: \url{http://www.ifa.hawaii.edu/faculty/barnes/software/}. The code is written in C and has been tested on Mac OSX systems and various Linux distributions. The sole prerequisite is the GNU Scientific Library (GSL), which can be obtained from \url{}{}. An \textit{Instructions} file with the distribution describes the installation process. The current version\footnote{As of 24 January 2014} follows.

\subsubsection{Installation Instructions}

Here's a BRIEF outline of how to install Zeno on a Mac under OS 10.6.  With trivial modifications, Zeno can also be installed under Linux or Unix; see the notes at the end for details.

This process could be streamlined, and I would appreciate any suggestions on ways to simplify the installation procedure.  However, this works...

\begin{enumerate}
  \item{The Gnu C Compiler (gcc) is recommended when installing and using
Zeno.  If you've not already done so, install gcc from the OS X Install
disk, or download it from www.gnu.org/software/gcc or a mirror site.}
  \item{Zeno now uses the Gnu Scientific Library (gsl), which is freely
available from www.gnu.org/software/gsl/ .  You will need to install GSL
before proceeding.  If you need to install it in a non-standard
location, see note d) below.}
  \item{Select a directory in which to install Zeno.  For simplicity, I'll
assume in the following that you are using your home directory, and
abbreviate it as ``~''.  In this directory, extract the Zeno archive: \\
  \begin{verbatim}
   tar -xvf zeno.VER.tar.gz
  \end{verbatim}
where VER is the version of the archive.  This will place the entire
distribution tree in directory ``~/zeno''.}
\item{If an update is available on the web site, extract it in the same
directory as the zeno archive:
  \begin{verbatim}
tar -xvf zeno.VER1.tar.gz
  \end{verbatim}
where VER1 is the version of the update.  This will replace updated
files with newer versions.}
  \item{Assuming you're using bash (the GNU Bourne-Again SHell), edit ``~/.bash\_profile'' and insert the following lines:
\begin{verbatim}
        export ZENOPATH="/Users/<yourname>/zeno"
        export ZCC="gcc"
        export ZCCFLAGS="-DMACOSX -I$ZENOPATH/inc"
        export ZLDFLAGS="-L$ZENOPATH/lib"
        export ZENO_SAFE_SELECT="true"
        export ZENO_MSG_OPTION="all"
\end{verbatim}
Here, ``<yourname>'' is your user name; this assumes (as per step \#2) that you decided to install zeno under your home directory.  If you installed it elsewhere, modify ZENOPATH accordingly.  You will probably also want to add \$ZENOPATH/bin to your search PATH.  Once you've made these changes, start a new window or execute ``source ~/.bash\_profile'' so they will be in effect for the next step.

The variables ``ZENO\_SAFE\_SELECT'' and ``ZENO\_MSG\_OPTION'' influence the behavior of Zeno software.  Undefining ``ZENO\_SAFE\_SELECT'' makes access to particle data slightly faster, but sacrifices detection of references to undefined data fields.  Setting ``ZENO\_MSG\_OPTION'' to ``warn'' limits informational messages to urgent warnings, while setting it to ``none'' suppresses even these.}
\item{Finally, change to ``~/zeno'' and build the system:
\begin{verbatim}
make -f Zeno install_all >& zenomake.log
\end{verbatim}
This takes about 28 seconds on a 2.4 GHz iMac; your mileage may vary.  Currently, the last line in zenomake.log shows several versions of sphcode being moved to the zeno bin directory; if the build got that far then chances are everything is OK!

If you examine zenomake.log, you may find warnings about comparisons or casts between pointers and integers, use of tempnam, etc.  None of these appear to indicate serious problems.  Eventually I hope to fix them.}
\end{enumerate}

\subsection{Summary of Zeno Programs}
\label{ssec:programs}

\emph{description of the naming convention}

Passing the argument ``-help'' to any zeno program will result in a short listing of the possible program arguments, while ``-explain'' will provide a slightly more verbose description, in some cases with a list of valid values.

%-------------------------------------------------------------------------------
\section{N-body Simulations}
\label{sec:Nbody}

Zeno uses a standard treecode scheme to compute the gravitational force \citep{Barnes1986}. 

%-------------------------------------------------------------------------------
\section{Smoothed Particle Hydrodynamics}
\label{sec:SPH}

\subsection{Star Formation}

Zeno implements star formation in a probabilistic fashion, consistent with the monte carlo nature of N-body simulations. 

%-------------------------------------------------------------------------------
\section{Creation of Initial Conditions}
\label{sec:IC}

Zeno features a variety of tools to construct galaxy models and place these models on encouters for 

%-------------------------------------------------------------------------------
\section{Simulation Analysis}
\label{sec:Analysis}

Zeno contains a suite of to assist in analysing the snapshots produced by simulations. These utilities can be combined in makefiles or scripts to build custom analysis pipelines.

\subsection{Tracking the galaxy centers}

\begin{verbatim}
cat s001b_0?00.dat | snapsift - - sieve='type==0x42' | snaptrak - centers.dat group='i%(n/2)<512 ? (i < n/2 ? 1 : 2) : 0'
\end{verbatim}

Breaking apart this example, the pipeline first loads a subset of snapshots with the s001b prefix, then extracts only those particles with a BodyType tag equal to 0x42 (in this case, corresponding to the stellar bulge). These bulge particles are then passed to the snaptrak program which computes the (x,y,z) coordinates as well as the 3D velocity and accelerations of the center. The group argument specifies that it should only include the first 512 particles from each bulge (the data file is set up so the most tightly bound particles are listed first in the array). The results are written to a Zeno structured file called ``centers.dat''. This file can then be converted to ascii using the \texttt{snapascii} program.

\subsection{Visualizing a Simulation Snapshot}

The \texttt{snapview} program takes an input snapshot and displays it in an interactive viewer. 

Multiple snapshots can be passed to \texttt{snapview} at a time; a single snapshot can be displayed and the user can step through the sequence of snapshots using the spacebar. This can be useful for obtaining a quick look at the dynamical evolution of the system. Color-coding can be added for either the line-of-sight doppler motions (dopcolor=true) or via the use of the ``colordata'' argument (e.g., ``colordata=uint'' utilizes the Uinternal values to colorize particles).

\end{document}
